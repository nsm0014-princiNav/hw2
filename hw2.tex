\documentclass[12pt,letterpaper, onecolumn]{exam}
\usepackage{amsmath}
\usepackage{amssymb}
\usepackage{graphicx}
\usepackage{setspace}
\usepackage{nicefrac}
\usepackage{hyperref}
\setcounter{MaxMatrixCols}{20}
\usepackage[lmargin=71pt, tmargin=1.2in]{geometry}  %For centering solution box
\lhead{Principles of Navigation}
\rhead{Noah Miller}
\thispagestyle{empty}   %For removing header/footer from page 1

\begin{document}

\begingroup
\centering
\LARGE Principles of Navigation\\
\LARGE Homework 2 \\[0.5em]
\large \today\\[0.5em]
\large Noah Miller\par
\large 903949330\par
\large MECH 6970\par
\endgroup
\pointsdroppedatright   %Self-explanatory
\printanswers
\renewcommand{\solution}{\noindent\textbf{Answer:}\enspace}   %Replace "Ans:" with starting keyword in solution box



\begin{questions}
    \question{Pedestrian Navigation System: Use your phone to implement a pedestrian navigation
        system. Estimate your position by starting at a known location and dead reckoning using
        an open-source step counter and compass. Make sure that your route is at least 1 km and
        contains turns in both directions. Use the compass to track your orientation. You can
        record measurements (i.e. step and compass reading) on your phone if you have that
        functionality. Otherwise use pad and paper to track steps and compass changes.

        You may use a default step length (as discussed in class) or try to estimate your step
        length (a priori) using a measuring instrument of your choice (e.g. tape measure or GPS).
        You should update position with each step and orientation at least every time it changes
        significantly. Plot your route on a map background using GPS visualizer
        (\url{https://www.gpsvisualizer.com/}) or a Matlab toolbox if available. Measure the error in
        your final position and orientation. Report total error and error relative to distance
        traveled.}
    \clearpage
    \question{Find a scholarly article describing a navigation system for a team of at least two ground
        vehicles. Describe the sensors used, states estimated, and estimator architecture. Give a
        brief description of the problem statement/motivation for the work. What results were
        presented (e.g. plots of positioning accuracy, raw sensor data, etc.)? Did the authors
        successfully convey the methodology?}

    \solution{%
        In "Corridors-based naviation for automated vehicles convoy in off-road enviroments", the authors (Godoy J., et al.) present a motion-planning architecture based on the wake generation of the leading vehicle to optimally estimate a navigable trajectory. Godoy employs this methodology becuase markings found on highways and roads are either hard to detect using optical sensors, or just do not exist at all when travelling off-road. A handful of sensors colloborate to estimate the trajectory. These include:
        \begin{itemize}
            \item Radar
            \item Camera
            \item LiDAR
            \item IMU
            \item GNSS receivers
        \end{itemize}
        A localization algorithm uses the sensors above to feed into a perception algorithm. The outputs are then communicated to each following vehicle to where the guidance module takes over to feed the determined trajectory. They estimate the vehicles' position in the global frame, as well as the velocity and orientation of each vehicle with respect to the body frame. They also track the distance between each vehicle and try to maintain less than 50 meters between each vehicle, although some experiments showed the convoy separate by at least 60 meters. The authors display several plots over the course of the paper including plots on waypoint generation, average convoy speed, post-processed trajectories, and error over time with regard to leader-follower difference in speed, north position, and east position.

        I think the authors accurately conveyed their approach. They provide a theoretical algorithm and overview of their off-road grid planning trajectory system and then provide data from field tests with references to the steps inline with the algorithm block diagram.

    }
    \clearpage
    \question{In class, we developed the basic (elementary) rotation matrix
        \begin{equation*}
            C_{z,\theta} =
            \begin{bmatrix}
                cos(\theta) & -sin(\theta) & 0 \\
                sin(\theta) & cos(\theta)  & 0 \\
                0           & 0            & 1 \\
            \end{bmatrix}
        \end{equation*}
        that describe the orientation of a coordinate frame rotated away from another coordinate frame by an angle $\theta$ about the $y$-axis.}
    \begin{parts}
        \part{Derive the basic (elementary) rotation matrix $C_{y,\theta}$ that describes the orientation of a coordinate frame rotated away from another coordinate frame by an angle $\theta$ about the $y$-axis.}

        \solution{%
            \begin{figure}[!h]
                \centering
                \includegraphics[width=0.5\linewidth]{Q3a.png}
                \caption{Standard 3-dimensional axis with an applied rotation, $\theta$, about the $y$-axis.}
                \label{fig:1}
            \end{figure}

            Using figure \ref{fig:1}, a $3 \times 3$ matrix comprised of dot products from each combination of unit vectors between the two frames and the rotation, $\theta$ will give us $C_{y,\theta}$.
            \begin{equation}
                \begin{split}
                    C_{y,\theta} & =
                    \begin{bmatrix}
                        x'x_1 & y'x_1 & z'x_1 \\
                        x'y_1 & y'y_1 & z'y_1 \\
                        x'z_1 & y'z_1 & z'z_1 \\
                    \end{bmatrix}\\
                    C_{y,\theta} & =
                    \begin{bmatrix}
                        cos(\theta) & 0 & -sin(\theta) \\
                        0           & 1 & 0            \\
                        sin(\theta) & 0 & cos(\theta)  \\
                    \end{bmatrix}\\
                \end{split}
                \label{eq:1}
            \end{equation}
        }
    \end{parts}
    \clearpage
    \question{For each of the matrices below, determine which are valid rotation matrices.  Justify
        your answer based upon expected properties.}
    \begin{parts}
        \part{
            \begin{equation*}
                c^a_b =
                \begin{bmatrix}
                    0  & 0 & 1 \\
                    0  & 1 & 0 \\
                    -1 & 0 & 0 \\
                \end{bmatrix}
            \end{equation*}
        }
        \part{
            \begin{equation*}
                c^b_c =
                \begin{bmatrix}
                    1 & 0 & -1 \\
                    0 & 1 & 0  \\
                    0 & 0 & 0  \\
                \end{bmatrix}
            \end{equation*}
        }
        \part{
            \begin{equation*}
                c^b_c =
                \begin{bmatrix}
                    1 & 0           & 0 \\
                    0 & \frac{1}{2} & 0 \\
                    0 & 0           & 2 \\
                \end{bmatrix}
            \end{equation*}
        }
        \part{
            \begin{equation*}
                c^d_e =
                \begin{bmatrix}
                    0.4330  & -0.7718 & 0.4656 \\
                    0.7500  & 0.5950  & 0.2888 \\
                    -0.5000 & 0.2241  & 0.8365 \\
                \end{bmatrix}
            \end{equation*}
            \part{
                \begin{equation*}
                    c^e_f =
                    \begin{bmatrix}
                        0.5000  & -0.1464 & 0.8536  \\
                        0.5000  & -0.8536 & -0.1464 \\
                        -0.7071 & 0.5000  & 0.5000  \\
                    \end{bmatrix}
                \end{equation*}
            }
        }
    \end{parts}
    \clearpage
    \question{Consider the rotation matrix
        \begin{equation*}
            C^0_1 =
            \begin{bmatrix}
                0  & -\frac{\sqrt{3}}{2} & -\frac{1}{2}       \\
                0  & -\frac{1}{2}        & \frac{\sqrt{3}}{2} \\
                -1 & 0                   & 0                  \\
            \end{bmatrix}
        \end{equation*}
        \begin{parts}
            \part{Sketch frames $0$ and $1$ with their origins co-located.}

            \part{Given a vector $\mathbf{v}^0 = \left[\;1 \quad 1 \quad 1\;\right]^T$ coordinated in frame $0$, re-coordinatize the vector such that it is described in frame 1.}
        \end{parts}}
    \clearpage
    \question{For each pair of coordinate frames, find the rotation matrix $C^a_b$ that describes their relative orientation.
        \begin{parts}
            \part{
                \begin{figure*}[!h]
                    \centering
                    \includegraphics[width=0.25\linewidth]{Q6a.png}
                \end{figure*}
            }
            \part{
                \begin{figure*}[!h]
                    \centering
                    \includegraphics[width=0.25\linewidth]{Q6b.png}
                \end{figure*}
            }
        \end{parts}}
    \clearpage
    \question{Coordinate frame $\{1\}$ is obtained from frame $\{0\}$ by the following sequence of rotations:
        \begin{itemize}
            \item[i.]{$-90^o$ about the fixed $z$-axis}
            \item[ii.]{$90^o$ about the current $y$-axis}
            \item[iii.]{$-90^o$ about the current $x$-axis}
        \end{itemize}
        Find the resulting rotation matrix $C^0_1$ and sketch frames $\{0\}$ and $\{1\}$ relative to each other.
    }
    \clearpage
    \question{Given the roll-pitch-yaw angles ($\phi$, $\theta$, $\psi$) = ($120^o$, $45^o$, $-120^o$), find the rotation matrix that describes the same orientation. Assume the $ZYX$ series of rotations. Verify that you have constructed the correct rotation matrix by backing out the angles.}
    \clearpage
    \question{(Required for 6790 only) Consider the threee coordinate frames $\{\alpha\}$, $\{\beta\}$, and $\{\gamma\}$ shown in the diagram below. Following the notation introduced in the class, find the following Cartesian position vectors (denoted by $\vec{r}$\;) and rotation matrices (denoted by $C$).
        \begin{figure*}[!h]
            \centering
            \includegraphics[width=0.85\linewidth]{Q9.png}
        \end{figure*}
        \clearpage
        \begin{parts}
            \part{$\vec{r}^{\gamma}_{\gamma\alpha}$}
            \part{$\vec{r}^{\gamma}_{\gamma\beta}$}
            \part{$\vec{r}^{\alpha}_{\gamma\alpha}$}
            \part{$\vec{r}^{\beta}_{\gamma\beta}$}
            \part{$\vec{C}^{\gamma}_{\alpha}$}
            \part{$\vec{C}^{\gamma}_{\beta}$}
            \part{$\vec{C}^{\alpha}_{\beta}$}
        \end{parts}
    }

\end{questions}
\end{document}x